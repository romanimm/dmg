\section{Lernziel E: Die Grundlagen von Datenbank-Systemen definieren}

\subsection{Die Grundbegriffe von Datenbank-Systemen erklären:}

\subsubsection{Datenbasis}

Als Datenbasis werden die gespeicherten Daten bezeichnet. Die Datenbasis enthält die miteinander in Beziehung stehenden Informationseinheiten.

\subsubsection{Zugriffsprogramme} 

Zugriffsprogramme werden entwickelt um den Benutzer einen einfachen Zugriff auf die Datenbasis zu ermöglichen.

\subsubsection{Datenbankverwaltungssystem}

Mit einem \ac{DBMS} wird der Zugriff auf die Datenbasis, die Kontrolle der Konsistenz und die Modifikation der Datenbasis ermöglicht. Beispiele: Microsoft SQL Server, CICS von IBM.

\subsection{Die Gründe für den Einsatz von Datenbanken beschreiben}

\begin{itemize}
	\item Redundanz und Inkonsistenz vermeiden
	\item Zugriffsmöglichkeiten beschränken
	\item Mehrbenutzerbetrieb ermöglichen
	\item Verlust von Daten verhindern
	\item Integritätsverletzungen verhindern
	\item Sicherheitsprobleme vermeiden
	\item Entwicklungskosten minimieren
\end{itemize}

\subsection{Die Datenmodellierung anhand der folgenden Begriffe erläutern:}

\subsubsection{Datenmodell}

Das Datenmodell stellt einem Werkzeuge zur Verfügung um die reale Welt in einem Datenbankverwaltungssystem abzubilden. Das Datenmodell kann wie eine Programmiersprache für Datenbanken verstanden werden. 

Das Datenmodell besteht aus einer \ac{DDL} und einer \ac{DML}. Mit der \ac{DDL} wird das Datenbankschema beschrieben. Also die Struktur der Daten (Welche Datentypen, Spalten usw.). Die \ac{DML} besteht aus einer Anfragesprache (z.B. \texttt{SELECT}) und einer Manipulationssprache (z.B. \texttt{INSERT INTO}). Es gibt auch Datenmodelle für den konzeptionellen Entwurf (z.B. ER-Modell)

\subsubsection{Datenbankschema}

Das Datenbankschema legt die Struktur der abgespeicherten Datenobjekte fest. So haben z.B. alle Studenten eine Matr.-Nr. und einen Namen (Datenbankschema). Aber jeder Student hat einen individuelle Matr.-Nr. und einen anderen Namen. Dieser aktuelle Zustand nennt man Datenbankausprägung. 

Die Datenbankausprägung wird ständig verändert wohingegen das Datenbankschema selten ändert. Eine Änderung am Datenbankschema kann schwerwiegende Folgen haben.