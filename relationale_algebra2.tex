\section{Lernziel R2: Operatoren der Relationalen Algebra anwenden}

\subsection{Der Student versteht wann und wie man übliche Mengenoperationen wie Vereinigung, Durchschnitt und Differenz auf Relationen übertragen kann}

\subsubsection{Vereinigung}

Sind $R$ und $S$ zwei typ-kompatible Relationen (gleiches Schema) so kann man ihre Tupel vereinigen. Man schreibt:

\begin{equation}
R \cup S
\end{equation}

\subsubsection{Durchschnitt}

Sind $R$und $S$ zwei typ-kompatible Relationen (gleiches Schema) so kann man ihre Schnittmenge bilden. Man schreibt:

\begin{equation}
R \cap S
\end{equation}

\subsubsection{Differenz}

Sind $R$ und $S$ zwei typ-kompatible Relationen (gleiches Schema) so sind im Ergebnis alle Tupel von R enthalten, die in S nicht enthalten sind bilden. Man schreibt:

\begin{equation}
R \setminus S
\end{equation}

\subsection{Der Student versteht wie man Relationen mittels Selektion und Projektion verkleinern kann}

\subsubsection{Selektion}

Bei der Selektion werden diejenigen Tupel ausgewählt welche die Bedingungen erfüllen (wie \verb|where| in SQL). Eine Auswahl filtert Zeilen heraus. Das Beispiel wählt alle Studenten aus welche länger als 10 Semester studiert haben:

\begin{equation}
\sigma_{Semester>10}(Studenten)
\end{equation}

\subsubsection{Projektion}

Mit einer Projektion können Spalten einer Tabelle ausgewählt werden. Das Beispiel gibt nur die Namen der Professoren aus:

\begin{equation}
\Pi_{Name}(Professoren)
\end{equation}

\subsubsection{Umbenennung}

Teilweise ist es notwendig eine Tabelle umzubenennen, weil sie z.B. zweimal vorkommt. Man schreibt:

\begin{equation}
\rho_{S}(R)
\end{equation}

\subsection{Der Student versteht, wie man zwei Relationen zu einer einzigen Relationen via Kreuzprodukt oder Join verknüpft und kennt dabei die Unterschiede zwischen verschiedenen Joins}

\subsubsection{Kreuzprodukt}

Beim Kreuzprodukt wird jede Zeile der Relation $R$ mit jeder Zeile der Relation $S$ kombiniert (Spezialfall von Join). Man schreibt:

\begin{equation}
R \times S
\end{equation}

\subsubsection{Join}

In diesem Abschnitt wird lediglich die Notation von Joins in der relationalen Algebra aufgezeigt. Wie die einzelnen Joins funktionieren ist dem Abschnitt \ref{sec:join} zu entnehmen.

\begin{itemize}
	\item Cross Join: $R \times S$
	\item Natural Join: $R \bowtie S$
	\item $\theta$-Join (Theta-Join): $R \bowtie_{PersNr=gelesenVon} S$
	\item Left Outer Join: $R \: {\tiny \textifsym{d|><|}} \: S$
	\item Right Outer Join: $R \: {\tiny \textifsym{|><|d}} \: S$
	\item Full Outer Join: $R \: {\tiny \textifsym{d|><|d}} \: S$
\end{itemize}

\subsection{Der Student kann den Divisionsoperator als Umkehrung des Verbunds (Join) anwenden}

Die Division lässt sich am einfachsten mit einem Beispiel erklären:

\begin{table}[!htp]
	\centering
	\begin{subtable}[b]{0.3\textwidth}
		\centering
		\begin{tabular}{|c|c|c|c|}
			\hline
			Vater & Mutter & Kind & Alter \\
			\hline
			Franz & Helga & Harald & 5 \\
			\hline
			Franz & Helga & Maria & 4 \\
			\hline
			Franz & Ursula & Sabine & 2 \\
			\hline
			Moritz & Melanie & Gertrud & 7 \\
			\hline
			\rowcolor{green} Moritz & Melanie & Maria & 4 \\
			\hline
			\rowcolor{green} Moritz & Melanie & Sabine & 2 \\
			\hline
			Peter & Christina & Robert & 9 \\
			\hline
		\end{tabular}
		\subcaption{Relation $R$}
	\end{subtable}
	\begin{subtable}[b]{0.3\textwidth}
		\centering
		\begin{tabular}{|c|c|}
			\hline
			Kind & Alter \\
			\hline
			Maria & 4 \\
			\hline 
			Sabine & 2 \\
			\hline
		\end{tabular}
		\subcaption{Relation $S$}
	\end{subtable}
	\begin{subtable}[b]{0.3\textwidth}
		\centering
		\begin{tabular}{|c|c|}
			\hline
			Vater & Mutter \\
			\hline
			Moritz & Melanie \\
			\hline
		\end{tabular}
		\subcaption{Ergebnis der Division $R \div S$}
	\end{subtable}
	\caption{Divisionsoperator}
\end{table}

Moritz und Melanie sind die einzigen Eltern, die ein Kind Maria (4 Jahre) und ein Kind Sabine (2 Jahre) haben. Der Rest der Eltern erfüllt die Division nicht.

\subsection{Der Student kann Relationenalgebraische Ausdrücke umrechnen oder äquivalent darstellen}

Alle relationalalgebraischen Ausdrücke lassen sich mit folgenden Ausdrücken:

\begin{itemize}
	\item $R \cup S$
	\item $R \setminus S$
	\item $R \times S$
	\item $\sigma_p(R)$
	\item $\Pi_s(R)$
	\item $\rho_v(R)$
\end{itemize}

Die restlichen Bezeichnungen (Join, Division) wären nicht nötig und lassen sich durch die obengenannten Ausdrücke ersetzen.