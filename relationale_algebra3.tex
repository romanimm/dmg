\section{Lernziel R3: Zusammenhang Relationen und SQL}

\subsection{Der Student versteht, dass Tabellen einerseits Äquivalenzklassen von Relationen sind}
Bei Relationen ist klar, dass man die Reihenfolge der Elemente innerhalb eines Tupels nicht vertauschen darf.

\begin{equation*}
  (1,2,3) \neq (3,2,1)
\end{equation*}

In einer Tabelle sind die Spalten mit einem eindeutigen Namen versehen, deswegen darf man dies dort auch. Dann nehmen wir eine Äquivalenzrealation und teilen damit die Menge der Relationen in Äquivalenzklassen. Jede Vertauschung der Spalten, die damit eine neue Relation erstellt, gehört zur selben Äquivalenzklasse. Damit ist gezeigt, dass Tabellen ebenfalls Relationen sind.

\emph{Meinung des Autors:} Ich glaube, das ist gar nicht so wichtig... einfach wissen, dass Tabellen halt eigentlich Relationen sind, und dass die Operationen, welche auf die Relationen gelten, daher auch auf die Tabellen gelten.

\subsection{Der Student versteht, dass Tabellen andererseits Mengen von Funktionen sind}
Tabellen haben ein Schema mit Attributnamen. z.B. ID, Name, Vorname, Kategorie. Zu diesen Attributnamen gehört nun auch ein Wertebereich (=Domäne). z.B.

\begin{center}
  dom(ID) = \text{'Positive Zahl mit max. 10 Stellen'} \\
  dom(Name) = \text{'Alle Buchstaben'} \\
  dom(Kategorie) = \{\text{Student}, \text{Dozent}, \text{Assistent}\}
\end{center}

Tief durchatmen; Eine Funktion ordnet jedem Element aus einer Menge \(A\) ein Element aus einer anderen Menge \(B\) zu. Menge \(A\) ist bei uns die Menge der Attribute (das Schema von oben) und die Menge \(B\) die Domäne des entsprechenden Attributs.
Also: Jede Zeile ordnet jedem Attribut ein Element aus seiner Domäne zu. Daher:

\begin{center}
	\textbf{Jede Zeile der Tabelle ist eine Funktion.}
\end{center}

\subsection{Der Student kennt die Regeln, um relational-algebraische Ausdrücke äquivalent umzuformen}
\label{sec:umformung}

\subsubsection{Selektionsoperator}
(Zur Erinnerung; der Operator lässt die Zeilen/Tupel durch, welche gemäss dem Ausdruck \texttt{true} sind).

Eine konjuktive Selektion (in SQL wäre das \texttt{WHERE x=1 AND y=2 AND ...} kann aufgebrochen werden in eine Folge individueller Selektionsoperatoren.
\begin{equation}\label{konjuktive_selektion}
  \sigma _{c_1\wedge c_2 \wedge \dots \wedge c_n} (R) = \sigma _{c_1} (\sigma _{c_2}(\dots \sigma _{c_n} (R) \dots ))
\end{equation}

Das heisst, die Selektionsoperatoren können auch vertauscht werden (klar, können \texttt{AND} Operatoren in SQL ja auch...)

\begin{equation}
  \sigma _{c_1} (\sigma _{c_2} (R)) =  \sigma _{c_2} (\sigma _{c_1} (R)) 
\end{equation}

\subsubsection{Projektion}
Wenn wir eine Folge von Projektionen haben, können wir alle wegstreichen, bis auf die letzte.

\begin{equation*}
  \Pi _{List_1} ( \Pi _{List_2} ( \dots \Pi _{List_n} (R))) = \Pi _{List_1} (R)
\end{equation*}

\subsubsection{Projektion und Selektion}
Wenn die Attribute der Projektion und der Selektion gleich sind, ist die Projektion und Selektion austauschbar.

\subsubsection{Join Operator / Kreuzprodukt}
Der Join Operator und Kreuzprodukt sind kommutativ.

\begin{equation*}
  R \bowtie S = S \bowtie R
\end{equation*}
\begin{equation*}
  R \times S = S \times R
\end{equation*}

Wenn über ein Join ein Selektionsoperator gelegt wird, kann dieser, wenn er nur eine Tabelle betrifft, auch in den Join hinein versetzt werden.

\begin{equation}
  \sigma _c (R \bowtie S) =  (\sigma _c  (R))\bowtie S
\end{equation}

Sollte der Selektionsoperator beide Tabellen betreffen, also \(c = c_1 \wedge c_2\) dann kann je ein Selektionsoperator über die Tabelle gelegt werden.

\begin{equation}
  \sigma _c (R\bowtie S) = (\sigma _ {c_1} (R)) \bowtie (\sigma _ {c_2} (S))
\end{equation}

\subsubsection{Projektion und Join}
Wenn zuerst zwei Tabellen mit einem Join verbunden werden und dann noch eine Projektion (d.h. Spalten ausgeblendet) drüber gelegt wird, dann kann man einfach die Projektion vor dem Join bereits machen. Sofern dann aber nicht über genau diese Spalte gejoint wird, sonst geht ja das joinen nicht mehr...

\begin{equation}
  \Pi _{L} (R \bowtie _C S) = (\Pi _{A} (R)) \bowtie _C (\Pi _B (S))
\end{equation}

\subsubsection{Mengenoperationen}
Vereinigung \(\cup\) und Durchschnitt \(\cap\) sind kommutativ (= Seiten sind austauschbar).

Operationen wie \(\bowtie\) , \(\times\) , \(\cap\) und \(\cup\) sind assoziativ, das heisst in einem Fall wie

\begin{equation}
  A \circledast (B \circledast C)
\end{equation}

können die Klammern beliebig gesetzt werden.

\begin{equation}
  A \circledast (B \circledast C) = (A \circledast B) \circledast C
\end{equation}


Wenn wir eine Projektion oder eine Selektion mit Mengenoperationen haben, können wir die Projektion entweder über die Mengenoperationen legen, oder einzeln anwenden.

\subsection{Der Student versteht, wie relational-algebraische Ausdrücke als Baum dargestellt werden können}
\label{sec:anfragebaum}

Die Ausdrücke können ganz einfach in eine Baumform gebracht werden. Man nimmt einfach als Blatt die Tabelle und verbindet die Blätter von unten nach oben mit entsprechenden Operatoren, als Beispiel siehe Abbildung \ref{fig:querytree}.

\mygraphics{0.3\textwidth}{fig/query_tree.png}{Abfragebaum}{querytree}

\subsection{Der Student versteht, wie relational-algebraische Ausdrücke in Baumdarstellung in effizientere Formen umgeformt werden können}

Es sollen folgende Regeln angewendet werden:

\begin{itemize}
  \item Grundsatz: Zuerst sollen Operationen ausgeführt werden, welche die Zwischenresultate verkleinern.
  \item Regel aus Gleichung \ref{konjuktive_selektion} benützen um einzelne Selektionsoperatoren zu erhalten.
  \item Selektionsoperator so weit wie möglich runter bringen unter Anwendung der Regeln in Abschnitt \ref{sec:umformung}
  \item Die restriktivsten Selektionsoperatoren so früh wie möglich ausführen.
  \item Gruppen von Operationen können evt. zusammengefasst werden z.B. Selektion von einem Kreuzprodukt mit anschliessender Projektion ist einfach ein Join.
  \item Projektionsoperationen (Attribute selektieren) ebenfalls möglichst früh
\end{itemize}

\subsection{Der Student kennt einige Erweiterungen der Relationenalgebra}
Eigentlich braucht SQL gar keine relationale Algebra (WTF gell). Sondern sie brauchen die \emph{Bag-Algebra}. Denn in der relationalen Algebra darf es keine identischen Tupel in derselben Relation geben, in SQL offenbar ja schon. Also ein \emph{Bag} ist einfach eine Menge, in der die Elemente mehrfach vorkommen dürfen. Ebenfalls kommt es nicht auf die Reihenfolge drauf an. Operationen wie Selektion / Projektion löschen diese Duplikate auch nicht.

S \(\cup\) S = S gilt nicht bei \emph{Bags}.