\section{Lernziel S1: Die grundlegenden Konstrukte von SQL auf einem
Datenbankserver anwenden}

\subsection{Schemadefinition (create table)}

Listing \ref{lst:create-table} zeigt wie man eine Tabelle anlegt.

\begin{lstlisting}[caption={create table}, label=lst:create-table]
create table Professoren (
	-- Dieses Attribut darf nicht leer sein
	PersNr integer not null,
	Name varchar(10)
);
\end{lstlisting}

\subsection{Schemaveränderung (alter table)}

Listing \ref{lst:alter-table} zeigt wie eine Tabelle verändert werden kann.

\begin{lstlisting}[caption={alter table}, label=lst:alter-table]
alter table Professoren add (
	-- Spalte Raum zur Tabelle hinzufügen
	Raum integer
);
\end{lstlisting}

\subsection{Einfügen von Datensätzen (insert into)}

Listing \ref{lst:insert-into} zeigt wie ein Tupel in die Tabelle eingefügt wird.

\begin{lstlisting}[caption={insert into}, label=lst:insert-into]
-- ohne Spaltennamen
insert into Professoren values (2136, 'Curie', 'C4');
-- mit Spaltennamen
insert into Professoren (PersNr, Name, Raum) values (2136, 'Curie', 'C4');
\end{lstlisting}

\subsection{Einfache Abfragen (select)}

Listing \ref{lst:select} zeigt unterschiedliche Abfragen.

\begin{lstlisting}[caption={select}, label=lst:select]
-- alle Attribute von Professor Curie
select * from Professoren where Name = 'Curie';
-- nur den Raum von Professor Curie
select Raum from Professoren where Name = 'Curie';
-- Alle Professoren nach Namen sortiert (A/1 kommt zuerst)
select * from Professoren order by Name asc;
-- Alle Professoren nach Namen sortiert (Z/9 kommt zuerst)
select * from Professoren order by Name desc;
-- distinct eleminiert Duplikate
select distinct Raum from Professoren; 
\end{lstlisting}

\subsection{Abfragen über mehrere Tabellen (,)}

Listing \ref{lst:multiple-tables} zeigt eine Abfrage über mehrere Tabelle. Über die \verb|where|-Bedingung können Tabellen wie bei einem Join verknüpft werden.

\begin{lstlisting}[caption={Abfragen über mehrere Tabellen}, label=lst:multiple-tables]
select p.Name, v.Titel from Professoren p, Vorlesungen v where p.PersNr = v.gelesenVon;
\end{lstlisting}

\subsection{Aggregatfunktionen und Gruppierung (avg, sum, count, min, max, group by)}

\subsubsection{Durchschnitt (avg)}

Die Funktion \verb|avg(column_name)| berechnet den Durchschnitt von numerischen Werten einer Spalte.

\subsubsection{Summe (sum)}

Die Funktion \verb|sum(column_name)| berechnet die Summe von numerischen Werten einer Spalte.

\subsubsection{Zusammenzählen (count)}

Die Funktion \verb|count(column_name)| zählt die Anzahl der Einträge einer Spalte (\verb|null| wird nicht gezählt). Mit der Anweisung \verb|count(*)| wird die Anzahl Ergebnisse einer Anfrage zurückgeliefert.

\subsubsection{Minimum (min)}

Die Funktion \verb|min(column_name)| gibt den kleinsten Wert einer Spalte zurück (Bei Zeichenketten wird bei A begonnen).

\subsubsection{Maximum (max)}

Die Funktion \verb|max(column_name)| gibt den grössten Wert einer Spalte zurück (Bei Zeichenketten wird bei Z begonnen).

\subsubsection{Gruppieren (group by)}

Bei der Funktion \verb|group by column_name| werden die Ergebnisse einer Spalte gruppiert (zusammengefasst).