\section{Lernziel OM: Mit Objektrelationalem Mapping Tabellen als Objekte lesen und schreiben}

\subsection{Die Vorteile von objekt-orientierten Datenbanken (OO-DB) erklären: Methoden, Referenzierung mit Pointer, many-to-many-Relationships, Vererbung}

\subsubsection{Methoden}
Objekte in OO-DBs können ebenfalls Methoden besitzen wie in anderen Programmiersprachen. Diese Methoden kapseln den Datenzugriff und können auch komplexere Operationen direkt kapseln. In relationalen Datenbanken findet man dies nicht.

\subsubsection{Referenzierung mit Pointer}
Joins benötigt es in OO-DBs nicht mehr. Verwandte Objekte werden direkt über die im Objekt gespeicherte Beziehung angesprochen.

\subsubsection{many-to-many-Relationships}
Die N:M-Beziehungen benötigen im relationalen Umfeld immer eine Zwischentabelle. In der OO-DB Welt benötigt es diese nicht mehr, da verwandte Objekte direkt im Objekt als Sammlung von Referenzen abgelegt sind.

\subsubsection{Vererbung}

Vererbung (Generalisierung) kann in relationalen Datenbanken nur schlecht abgebildet werden (Abschnitt \ref{sec:generalisierung}). In OO-DBs kann man Vererbung direkt modellieren.

\subsection{Die Vorteile von objektrelationalen Mappern erklären: Basis RDBMS (Investitionsschutz), Verwendung in OOP}

\subsubsection{Basis RDMBS (Investitionsschutz)}
Die weit verbreiteten RDBMS müssen nicht neu entwickelt werden, sie sind stark ausgereift und bereits getätigte Investitionen (Ausbildung, Lizenzen, Hardware) können so geschützt werden.

\subsubsection{Verwendung in OOP}
In der OOP verhalten sich die Daten einer Datenbank wie normale Objekte. Eine Middleware abstrahiert den Datenbankzugriff und die Speicherung der Objekte.

\subsection{Daten mit ORM (JPA) anfragen (JPQL) und in Java verwenden}

JPA ist die Middleware zwischen relationalen Datenbanken und der objektorientierten Java Welt. Sie besteht aus folgenden Komponenten:

\begin{itemize}
  \item \textbf{EntityManager} \\
  Objekt für den Zugriff auf persistente Objekte. Bietet Funktionen wie suchen, speichern oder Transaktionen an.
  \item \textbf{Entity} \\
  Ein einfaches Java Objekt, welches in einer Tabelle in der Datenbank gespeichert wird.
  \item \textbf{Annotations} \\
  Diese @-Symbole und Texte, die den Bezug von Instanzvariablen zu Spalten und von Klassen zu Tabellen herstellen.
  \item \textbf{JPQL}  \\
  Ähnlich wie SQL. Erlaubt es, Objekte von der Datenbank abzufragen.
  \item \textbf{persistence.xml} \\
  Herzstück von JPA - hier werden die Verbindungsdaten gespeichert.
\end{itemize}

\newpage

Listing \ref{lst:jpa-abfrage} zeigt einige Beispielanfragen mittels JPA.

\begin{lstlisting}[language=Java,keywordstyle=\color{keywordcolor},caption={JPA Grundlagen in Java},label=lst:jpa-abfrage]
// EntityManager erstellen
EntityManagerFactory emf = Persistence.createEntityManagerFactory("DMG_OOPU");
EntityManager em = emf.createEntityManager();

// Alle Professoren abfragen und in Liste speichern
Query query = em.createQuery("select p from Professoren p");
List list = query.getResultList();

// NamedQuery-Abfrage (Wird mit Annotations definiert)
Query q = em.createNamedQuery("Professoren.findAll");
List list = q.getResultList();
\end{lstlisting}

\subsection{Daten in Java erstellen und in die Datenbank schreiben (persist, merge)}

\texttt{persist} speichert ein neues Objekt in der Datenbank und \texttt{merge} aktualisiert ein vorhandenes Objekt. Beide Aktionen müssen innerhalb einer Transaktion ausgeführt werden. Listing \ref{lst:jpa-erstellen} zeigt wie ein Objekt in JPA erstellt und aktualisiert wird.

\begin{lstlisting}[language=Java,keywordstyle=\color{keywordcolor},caption={Daten updaten mit JPA},label=lst:jpa-erstellen]
// Transaktion beginnen
EntityTransaction ta = em.getTransaction();
ta.begin();

// Neuer Professor erstellen
Professoren p = new Professoren("Precht", "C3", 1337, 6565);

// Professor abspeichern
em.persist(p);

// Professor aus der DB holen
Query q = em.createQuery("SELECT p FROM Professoren AS p WHERE p.name = 'Precht'", Professoren.class);
Professoren s = (Professoren) q.getSingleResult();

// Professor aktualisieren
s.setRaum(1234);

// Aktualisierten Professor in DB schreiben
em.merge(s);

// Transaktion abschliessen
ta.commit();
\end{lstlisting}