\section{Lernziel DI: Semantische Integrität einer Datenbank sicherstellen}

\subsection{Den Begriff der referentiellen Integrität erklären (Kapitel 5.1)}
\subsubsection{Referentielle Integrität}
Ein Primärschlüssel einer Tabelle \(R\) ist in einer anderen Tabelle \(S\) als \(a\) enthalten. Man nennt \(a\) einen Fremdschlüssel wenn folgende zwei Bedingungen erfüllt sind:

\begin{itemize}
  \item \(a\) ist entweder immer \texttt{null} oder überhaupt nie \texttt{null}.
  \item Wenn \(a\) nicht \texttt{null} ist, dann muss jeder Wert von \texttt{a} in der anderen Tabelle ebenfalls vorkommen.
\end{itemize}
Wenn diese Bedingungen gegegeben sind, nennt man das eine \emph{referentielle Integrität}. 

Ein Vorteil der Integritätsprüfung auf der Datenbank ist der, dass man sie nur einaml zu implementieren braucht. Bei neuen Anforderungen muss die Applikationssoftware nicht angepasst werden, da sie zentral formuliert wird. (Anmerk. des Autors: Wohl kaum).

\subsubsection{Arten der Integritätsbedingungen}
\begin{itemize}
    \item \textbf{Statische Integritätsbedingung} \\
    Muss in jedem \emph{Zustand} der Datenbank erfüllt sein. Das heisst, z.B. der Rang der Professors muss in jedem Fall \texttt{C1, C2, \dots } sein.
    \item \textbf{Dynamische Integritätsbedingung} \\
    Muss von jeder \emph{Änderung} der Datenbank erfüllt werden. Als Beispiel darf ein Professor nur immer befördert, nie aber degradiert werden.
    \item \textbf{Implizite Integritätsbedingung}
    Ist durch das Schema der Datenbank vorgegeben, beispielsweise durch die referentielle Integrität oder Attributsdomänen (z.B. max 64 chars).
\end{itemize}


\subsection{Referentielle Integritätsbedingungen in SQL formulieren (Kap. 5.3; Übung DI)}

\subsubsection{Bei Erstellung der Tabelle}
\begin{lstlisting}[caption={Referentielle Integrität bei Erstellung}]
    CREATE TABLE Bestellungen
    (
        BestellungID INT NOT NULL PRIMARY KEY,
        KundenID INT,
        CONSTRAINT kunde_kauft FOREIGN KEY (KundenID) REFERENCES Kunden (KundenID)
        ON UPDATE CASCADE
    );
\end{lstlisting}

\subsubsection{Nach der Erstellung der Tabelle}
\begin{lstlisting}[caption={Referentielle Integrität Update}]
    ALTER TABLE Bestellungen
    ADD CONSTRAINT kunde_kauft FOREIGN KEY (KundenID) REFRENCES Kunden
    ON UPDATE CASCADE
\end{lstlisting}

Eigentlich ganz easy.

\subsubsection{Folgen der Veränderung bei der referentiellen Integrität}
Was passiert wenn ein Fremdschlüssel verändert wird oder gar gelöscht wird? Dies wird durch folgende Angaben gesteuert:

\begin{itemize}
  \item \textbf{ON UPDATE} \\
  Beim Verändern des Fremschlüssels
  \item \textbf{ON DELETE} \\
  Beim Löschen des Tuples
\end{itemize}
Es können folgende Aktionen definiert werden:
\begin{itemize}
  \item \textbf{CASACDE} \\
  Führt die Änderung in den referenzierenden Tabellen weiter. Vorsicht! Kanne einen ganzen "Rattenschwanz" an Änderungen nach sich ziehen.
  \item \textbf{SET NULL} \\
  Kann beim Löschen eingsetzt werden, setzt die referenzierenden Fremschlüssel auf \texttt{NULL} wenn ein referenziertes Tupel gelöscht wird.
  \item \textbf{RESTRICT} \\
  Verhindert die Änderung / Löschung.
\end{itemize}


\subsection{Statische Integritätsbedingungen formulieren (Kap. 5.4, 5.6)}
Statische Integritätsbedingungen erfasst man mit dem \texttt{check}-Operator. Dabei können ganz beliebige Operationen ausgeführt werden, es wird am Schluss einfach überprüft, ob die angegebene Operation \texttt{true} oder \texttt{unknown} zurückgibt - in beiden Fällen wird die Aktualisierung / der neue Datensatz zugelassen.
\begin{lstlisting}[caption={Beispiel mit check}]
CREATE TABLE Studenten
(
    MatrNr    INTEGER PRIMARY KEY,
    Name      VARCHAR(30) NOT NULL,
    Semester  INTEGER CHECK (Semester BETWEEN 1 AND 13)
);
\end{lstlisting}

Man kann auch solche checks über mehrere Spalten machen mit komplexeren Abfragen und diese auch noch benennen. Dies macht man so;

\begin{lstlisting}[caption={Beispiel mit check und Constraint}]
CREATE TABLE Studenten
(
    MatrNr    INTEGER PRIMARY KEY,
    Name      VARCHAR(30) NOT NULL,
    Semester  INTEGER
    CONSTRAINT test_bedingung CHECK (...SQL...)
);
\end{lstlisting}

\subsection{Den Begriff der Trigger erklären (Kap. 5.7)}
Trigger sind Prozeduren, die zu vordefinierten Zeitpunkten auf der Datenbank ausgeführt werden, analog zu den Stored Procedures. Mit Triggern können ebenefalls Integritätsbedingungen forciert und überprüft werden.

Mögliche Aktionen, die ein Trigger auslösen;
\begin{itemize}
  \item insert
  \item update
  \item delete
\end{itemize}

Trigger können vor und/oder nach diesen Aktionen ausgeführt werden (\texttt{before} und \texttt{after}). In den Prozeduren selbst kann auf die eingefügten / geänderten (neu und alt!) / gelöschten Tupeln Bezug genommen werden sowie können diese auch verändert werden.

\subsection{Trigger auf der Datenbank implementieren (Kap. 5.7; Übung DI)}

\begin{lstlisting}[caption={Trigger Beispiel SQL}]
CREATE TRIGGER keineDegradierung
BEFORE UPDATE ON Professoren
FOR EACH ROW
    WHEN (old.Rang IS NOT NULL) BEGIN
    IF (old.Rang = 'C3' AND new.Rang = 'C2') THEN
           new.Rang = old.Rang;
                /*usw*/
    END IF
END;
\end{lstlisting}

Die diversen Optionen aus dem vorigen Kapitel können natürlich an den entsprechenden Stellen verändert werden. Dies sollte selbstverständlich sein.
