\section{Lernziel S2: Fortgeschrittene Konstrukte von SQL auf einem Datenbankserver anwenden}

\subsection{Geschachtelte Anfragen (inline views)}

SQL Abfragen können geschachtelt werden. Dabei werden 2 Unterarten unterschieden:

\begin{itemize}
  \item \textbf{Rückgabe besteht aus einem Tupel mit einem einzigen Wert} \\
  Das könnte z.B. sein, wenn man ein einfaches \texttt{SELECT COUNT(*) FROM Studenten} ausführt. \\
  Kann dort eingefügt werden, wo ein skalarer Wert erwartet wird.
  \item \textbf{Rückgabe besteht aus mehreren Tupeln} \\
  Dies ist ein normaler Select, z.B. \texttt{SELECT * FROM Studenten} \\
  Wird häufig in Kombination mit Mengenoperatoren eingesetzt (siehe Kapitel \ref{sec:mengen_operatoren}).
\end{itemize}

Nun kann es sein, dass in der Unterabfrage ein Attribut der Oberabfrage referenziert wird. Als Beispiel:

\begin{lstlisting}[caption={Korrelierte Unterabfrage},label=lst:unterabfrage_korreliert]
SELECT PersNr, Name,
	(SELECT sum(SWS) AS Lehrbelastung FROM Vorlesungen WHERE gelesenVon = PersNr)
FROM Professoren;
\end{lstlisting}

Hier wird in der Unterabfrage das Attribut PersNr von der Oberabfrage verwendet. Dies ist möglichst zu vermeiden, da für jedes auszuwertende Tupel der Oberabfrage die Datenbank auch alle Unterabfragen auswerten muss. Dies könnte erhebliche Perfomanceeinbusse zur Folge haben. Oft kann man dieses Problem aber mit einem einfachen Join oder einer Aggregatsfunktion umgehen.

\subsubsection{Temporäre Sichten}
Mit dem Operator \texttt{WITH} kann sehr einfach eine temporäre Relation definiert werden, die wie eine normale Tabelle anschliessend abgefragt werden kann. Dies stellt eine lesbarere Alternative zu Verschachtelten Abfragen dar.

\begin{lstlisting}[caption={with Operator},label=lst:with]
WITH abfrage AS (/* SQL Abfrage hier */);
SELECT * FROM abfrage;
\end{lstlisting}

\subsection{Mengen-Operatoren (union, in)}
\label{sec:mengen_operatoren}

\subsubsection{union}
Beispiel für eine Union Abfrage:

\begin{lstlisting}[caption={Union Abfrage Beispiel},label=lst:union_bsp]
(SELECT Name FROM Assistenten) UNION (SELECT Name FROM Professoren);
\end{lstlisting}

Fügt einfach 2 Abfrageergebnisse zusammen. Die Ergebnisse müssen übereinstimmende Attribute haben. Zudem werden Duplikate automatisch entfernt. Mit \texttt{UNION ALL} können die Duplikate beibehalten werden.

\subsubsection{in}
\label{sec:in_operator}

Mit dem \texttt{in} Operator wird einfach getestet ob ein Element Teil einer anderen Menge ist. Einfaches Beispiel:
\begin{lstlisting}[caption={Beispiel in Operator},label=lst:in_bsp]
SELECT Name FROM Professoren WHERE PersNr IN (SELECT gelesenVon FROM Vorlesungen);
\end{lstlisting}

Es kann auch eine konkrete Menge angegeben werden:

\begin{lstlisting}[caption={Diskrete Menge IN}]
SELECT * FROM Studenten WHERE Semester IN (1,2,3,4);
\end{lstlisting}

Mit dem vorangestellten \texttt{NOT} kann das Gegenteil gezeigt, werden, also dass ein Element \textit{nicht} Teil dieser Menge ist.

\subsection{Quantifizierte Anfragen (exists)}

\texttt{exists} überprüft, ob eine Unterabfrage leer ist oder nicht. Wenn die Menge leer ist liefert \texttt{exists} \texttt{false} und sonst \texttt{true}.

\begin{lstlisting}[caption={SQL Beispiel mit Exists},label=lst:bsp_exists]
SELECT Name FROM Professoren
WHERE EXISTS (SELECT * FROM Vorlesungen WHERE gelesenVon = Professoren.PersNr);
\end{lstlisting}

Diese Abfrage gibt alle Namen der Professoren zurück welche eine Vorlesung lesen.

\subsection{Nullwerte (null)}

\texttt{null} bedeutet in SQL, dass der Wert unbekannt ist, das heisst er ist nicht \texttt{1} oder \texttt{"Haus des Nikolaus"} oder gar \texttt{0}, er ist einfach unbekannt. Folgende Regeln erklären das genauer:

\begin{itemize}
  \item \textbf{Arithmetische Ausdrücke} \\
  \texttt{1 + null} oder \texttt{0 * null} wird zu \texttt{null}
  \item \textbf{Boolsche Ausdrücke} \\
  SQL kennt nicht nur \texttt{true} und \texttt{false} sondern auch \texttt{unknown}. Daher kann z.B. der Ausdruck \texttt{WHERE PersNr = 3} die Ergebnisse \texttt{true} und \texttt{false} haben, aber wenn \texttt{PersNr = null} ist, dann ist das Resultat der Überprüfung \texttt{unknown}.
  \item \textbf{Logische Ausdrücke} \\
  \texttt{not}: \texttt{unknown} bleibt \texttt{unknown}. \\
  \texttt{and}: \texttt{unknwon and true = unknown} \\
  \texttt{or}: \texttt{unkown or false = unknown} \\
  Dies sind die Randfälle. Wenn bei einem \texttt{or} eine Seite ja schon \texttt{true} ist, dann ist es ja sowieso schon \texttt{true}. Analog wenn bei einem \texttt{and} die eine Seite \texttt{false} ist, bleibt es sowieso immer \texttt{false}.
  \item \textbf{where Klauseln} \\
  Bei einem \texttt{where} werden Tuplen nur immer dann weitergereicht, wenn die Bedingung \texttt{true} ist. Sie werden bei \texttt{false} oder \texttt{unknown} nicht weitergereicht.
  \item \textbf{Gruppierungen} \\
  \texttt{null} zählt bei Gruppierungen als eigenständiger Wert und ist als solcher aufgezählt.
\end{itemize}

\subsection{Vergleichsoperatoren (between, in, like, case when)}
\subsubsection{between}
Folgende Ausdrücke sind gleichwertig:

\begin{lstlisting}[caption={Beispiel between}]
SELECT * FROM Studenten WHERE 1 <= Semester AND Semester <= 4
        
SELECT * FROM Studenten WHERE Semster BETWEEN 1 AND 4;
\end{lstlisting}

\subsubsection{in}
Siehe Abschnitt \ref{sec:in_operator}.

\subsubsection{like}
Der \texttt{like}-Operator wird häufig in Verbindung mit Suchen von Text eingesetzt. Dabei können Platzhalter eingesetzt werden für unbekannte Teiltexte.
\begin{itemize}
  \item \textbf{\%} Beliebig viele Zeichen
  \item \textbf{\underline {{ }{ }}} Genau ein Zeichen
\end{itemize}

\begin{lstlisting}[caption={Beispiel für like Operator}]
SELECT * FROM Studenten WHERE Name LIKE '%im_n%'
\end{lstlisting}

\subsubsection{case when}
Dies ist am Besten an einem Beispiel erklärt:

\begin{lstlisting}[caption={Beispiel für case when}]
SELECT Name, (case when Geschlecht = 1 then 'M'
                   when Geschlecht = 0 then 'W'
                   else 'unbekannt' end)
FROM Personen
\end{lstlisting}

Es wird das erste zurückgegeben, was \texttt{true} ist, der Rest wird nicht mehr angeschaut.

\subsection{Join-Syntax in SQL-92 (join, inner join, outer join)}
\label{sec:join}

\subsubsection{cross join}
Bedeutet das Kreuzprodukt.

\begin{lstlisting}[caption={Cross Join}]
SELECT * FROM Kunden CROSS JOIN Produkte;
\end{lstlisting}

\subsubsection{natural join}
Hier erfolgt der Join über die gleichen Spaltennamen. Sind keine gleichen Spaltennamen da, ist er gleichbedeutend mit einem Kreuzprodukt.

\begin{lstlisting}[caption={Natural Join}]
SELECT * FROM Kunden NATURAL JOIN Produkte;
\end{lstlisting}

\subsubsection{join / inner join}
Bedeutet dasselbe. Ist ein sog. \texttt{Theta-Join} aus der relationalen Algebra.

\mygraphics{0.2\textwidth}{fig/innerjoin.png}{Inner Join}{Inner Join}

Hier wird Tupel für Tupel verglichen. Wenn die zu vergleichenden Attribute in beiden Tabellen vorkommen, kommen sie in das Resultat. Gibt es keine Übereinstimmung, wird zur nächsten Zeile gewechselt.

\begin{lstlisting}[caption={JOIN ON Beispiel}]
SELECT * FROM Kunden
JOIN Produkte ON Produkte.KäuferID = Kunden.KundenID
\end{lstlisting}

\subsubsection{left outer join}
Gleich wie der inner join, allerdings gelangen alle Tupel der linken Tabelle in das Resultat, es könnte sein, dass \texttt{null}-Werte auf der rechten Seite sind. \texttt{left outer join} ist dabei gleichbedeutend wie \texttt{left join}.

\mygraphics{0.2\textwidth}{fig/leftjoin.png}{Left Join}{Left Join}

\begin{lstlisting}[caption={Left Join SQL}]
SELECT * FROM Kunden LEFT OUTER JOIN Produkte;
\end{lstlisting}

Hier würden nun alle Produkte ausgegeben, und wenn sich zu den Produkten ein Käufer fände, würde dieser mit angegeben. Wenn kein Käufer da wäre, würde \texttt{null} in den entsprechenden Attributen stehen.

\subsubsection{right outer join}
Analog zum left outer join, der right outer join.

\mygraphics{0.2\textwidth}{fig/rightjoin}{Right Join}{Right Join}

\begin{lstlisting}[caption={Right Join SQL}]
SELECT * FROM Kunden RIGHT OUTER JOIN Produkte;
\end{lstlisting}

\subsubsection{full join}
Hier werden einfach der \texttt{left outer join} und der \texttt{right outer join} zusammengefasst, denn es werden alle Datensätze aus beiden Tabellen sicher ausgegeben.

\mygraphics{0.2\textwidth}{fig/fulljoin.png}{Full Join}{Full Join}

\begin{lstlisting}[caption={Full Join SQL}]
SELECT * FROM Kunden FULL JOIN Produkte;
\end{lstlisting}

Hinweis: dies geht auf dem MySQL Server nicht. Left und Right Join müssen mit einem \texttt{UNION} verknüpft werden.

\subsection{Veränderungen am Datenbestand (update, delete)}

\subsubsection{update}
Um Zeilen zu verändern benötigt man das \texttt{update} Statement. Folgendes Beispiel zeigt die Syntax:

\begin{lstlisting}[caption={update Beispiel}]
UPDATE Studenten SET Semester = Semester + 1 WHERE MatrNr = 1337;
\end{lstlisting}

\subsubsection{delete}
Zum Zeilen löschen.

\begin{lstlisting}[caption={delete Beispiel}]
DELETE FROM Studenten WHERE MatrNr = 1337;
\end{lstlisting}

\texttt{delete} kann defaultmässig (zumindest in MySQL) nur ausgeführt werden, wenn in der Bedingung eine primäre Index Spalte vorhanden ist. Folgendes geht also nicht unbedingt:

\begin{lstlisting}[caption={delete ohne Index}]
DELETE FROM Studenten WHERE Semester > 10
\end{lstlisting}

\subsection{Sichten (create view)}
Sichten sind virtuelle Relationen und zeigen darin einen Ausschnitt des gesamten Modells. 

\begin{lstlisting}[caption={Beispiel für eine Sicht}]
CREATE VIEW PrüferHärte(Name, HärteGrad) AS (   
    SELECT Professoren.Name, AVG(prüfen.Note) FROM Professoren
    JOIN prüfen ON Professoren.PersNr = prüfen.PersNr
    GROUP BY Professoren.Name, Professoren.PersNR
)
\end{lstlisting}

Hier können auch die Spalten umbenannt werden. Sichten vereinfachen Abfragen auch, denn so kann direkt nach der PrüferHärte abgefragt werden, ohne die komplexere Abfrage kennen zu müssen.

