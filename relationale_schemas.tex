\section{Lernziel R: Ein konzeptuelles Datenmodell in ein relationales Schema umsetzen}

\subsection{Den mathematischen Formalismus der Relation und dessen Zusammenhang zur Schema-Definition nennen}

\subsubsection{Domänen}
Wir definieren Wertebereiche, z.B. a-z, 0-9, 0/1. Diese Wertebereiche können auch Domänen genannt werden. Wir schreiben Domänen als $D_{n}$

\subsubsection{Relation}
Beim kartesischen Produkt (Kreuzprodukt) werden alle Elemente der ersten Menge mit allen der zweiten Verbunden. Eine Relation $R$ ist definiert als eine Teilmenge des kartesischen Produkts der $n$ Domänen:

\begin{equation}\label{eq:relation-definition}
    R \subseteq D_{1} \times \dots \times D_{n}
\end{equation}

\subsubsection{Schema}
Das Schema einer Relation ist gegeben durch die $n$ Domänen, wie in Formel \ref{eq:relation-definition} gezeigt. Im Kontext von Datenbanken könnte man dies exemplarisch folgendermassen schreiben:

\begin{equation}\label{eq:relation-db-ex}
    Telefonbuch \subseteq string \times string \times integer
\end{equation}

\subsubsection{Tabelle}
Eine Tabelle besteht aus vielen Relationen, welche alle dieselben Attribute haben. Diese Attribute müssen eindeutig benennt sein.
Die Attribute $string$ oder $integer$ von der Formel \ref{eq:relation-db-ex} müssen einen eindeutigen Namen innerhalb der Tabelle besitzen. Dies wird mathematisch korrekt wie folgt dargestellt:

\begin{center}
Telefonbuch : \{[Name: string, Adresse: string, \underline{Telefonnummer:integer} ]\}
\end{center}

Die seltsamen geschweiften und eckigen Klammern haben einen Zweck. Die eckigen Klammern [ ] zeigen an, dass es sich um ein Tupel handelt, die geschweiften \{ \} dass es eine Menge ist. Es ist also eine Menge von Tupeln, wie das in Formel \ref{eq:relation-definition} schon definiert wurde. Die Unterstreichung des Attributs \(Telefonnummer\) bedeutet, dass dies der Primärschlüssel der Tabelle ist (Abschnitt \ref{sec:primary-key-def}).

\subsubsection{Ausprägung}
Die Ausprägung einer Relation ist durch die Teilmenge des Kreuzproduktes gegeben.

\subsubsection{Schlüssel}
\label{sec:primary-key-def}
Ein Schlüssel ist ein Attribut welches ein Tupel eindeutig identifiziert.

\subsection{Entitäten, Attributen, Beziehungen und Rollen in ein relationales Schema umsetzen}

\subsubsection{Entitäten}
Die Entitäten werden im relationalen Schema in Tabellen umgewandelt. 

\subsubsection{Attribute}
Die Attribute der Entitäten sind die Spalten der entsprechenden Tabellen.

\subsubsection{Beziehungen}
Eine Beziehung zwischen zwei Tabellen (Entitäten) kann allgemein durch eine Beziehungstabelle ausgedrückt werden. Die Beziehungstabelle enthält die Primärschlüssel beider Tabelle als Fremdschlüssel und evtl. noch weitere Attribute. Beispielsweise beim Kauf eines Produktes könnte die Beziehung zwischen \(Produkt\) und \(Kunde\) nebst den Schlüsselattributen \(ProduktID\) und \(KundeID\) ein beschreibendes Attribut \(Menge\) oder \(Kaufdatum\) enthalten. Bei 1:1- und 1:N-Beziehungen ist nicht immer eine Tabelle nötig (siehe Abschnitt \ref{sec:schema-zusammenfassen}).

\subsection{Relationale Schemas zusammenfassen (insbesondere 1:N; 1:1, Generalisierung, schwache Entitäten)}
\label{sec:schema-zusammenfassen}

\subsubsection{1:N Beziehungen zusammenfassen}
\label{sec:1-n-zusammenfassen}

Für 1:n Beziehungen schreibt man den Schlüssel der 1-Tabelle (\emph{Professoren}) als Fremdschlüssel in die n-Tabelle (\emph{Vorlesungen}). Der Fremdschlüssel in der Tabelle \emph{Vorlesungen} ist \emph{gelesenVon}.

\begin{center}
Professoren : \{[\underline{PersNr: int}, Name: string, Rang: string, Raum: int]\} \\
Vorlesungen : \{[\underline{VorlNr: int}, Titel: string, SWS, int, \emph{gelesenVon: int}]\} \\
\end{center}

\subsubsection{1:1 Beziehungen zusammenfassen}

Eine 1:1-Beziehung kann auf die gleiche Weise wie in Abschnitt \ref{sec:1-n-zusammenfassen} beschrieben zusammengefasst werden. Bei 1:N- und 1:1-Beziehungen sollten \texttt{NULL} vermieden werden. Als Beispiel sei genannt, dass z.B. jedes Land einen Präsidenten hat, und jeder Bürger kann nur Präsident von einem Land sein. Allerdings ist nicht jeder Bürger Präsident. Daher würde es keinen Sinn machen, bei der Tabelle \emph{Bürger} eine Spalte \emph{istPräsidentVon} einzufügen. Diese Spalte würde in die Tabelle \emph{Land} kommen.

\subsubsection{Generalisierung}

Eine Generalisierung (Vererbung) kann man darstellen indem der Primärschlüssel der Obertabelle als Fremdschlüssel in die Untertabelle eingetragen wird. Das nachfolgende Beispiel zeigt eine simple Umsetzung, wobei die Obertabelle \emph{Angestellte} zwei Untertabellen hat.

\begin{center}
Angestellte : \{[\underline{PersNr}, Name]\} \\
Professoren : \{[\underline{PersNr}, Rang, Raum]\} \\
Vorlesungen : \{[\underline{PersNr},Fachgebiet]\}
\end{center}

Das relationale Modell bietet sonst keine Möglichkeit die Vererbung darzustellen.

\subsubsection{Schwache Entitäten}

Schwache Entitäten können ohne eine starke Entität nicht existieren (z.B. Raum in Gebäude). Die schwachen Entitäten besitzen einen lokalen Schlüssel. Der lokale Schlüssel ist der Primärschlüssel der starken Entität (1:N-Beziehung wobei 1 die starke Entität ist). 

\subsection{Auf der Basis eines konzeptuellen Datenmodells ein relationales Schema in einem CASE-Werkzeug modellieren}

MySQL-Workbench ist ein Beispiel für ein \ac{CASE}-Werkzeug.