\section{Lernziel R: Ein konzeptuelles Datenmodell in ein relationales Schema umsetzen}

\subsection{Den mathematischen Formalismus der Relation und dessen Zusammenhang zur Schema-Definition nennen (Kapitel 3.1)}

\subsection{Entitäten, Attributen, Beziehungen und Rollen in ein relationales Schema umsetzen (Kapitel 3.2)}

\subsection{Relationale Schemas zusammenfassen (insbesondere 1:N; 1:1, Generalisierung, schwache Entitäten) (Kapitel 3.3)}

\subsection{Auf der Basis eines konzeptuellen Datenmodells ein relationales Schema in einem CASE-Werkzeug modellieren (Übung R)}