\section{Lernziel SA: Sicherheit von Daten in Datenbanken gewährleisten}

\subsection{Identifikation und Autorisierung in SQL umsetzen (create user)}

Die Funktion \texttt{password()} generiert einen Passwordhash. Dieser Hash kann dann verwendet werden um einen Benutzer zu erstellen. Listing \ref{lst:create-user} zeigt wie man einen Benutzer in MySQL erstellt.

\begin{lstlisting}[caption={Benutzer erstellen},label=lst:create-user]
-- Hash generieren
select password('abcdefg_12'); -- *58DDF826FB6D08F2D78CE691773F242F10C161A1
-- Benutzer erstellen
create user reader identified by password '*58DDF826FB6D08F2D78CE691773F242F10C161A1';
\end{lstlisting}

\subsection{Zugriffskontrolle auf Tabellen in SQL umsetzen (grant, revoke)}

Es existieren zwei Sicherheitsstrategien:

\begin{itemize}
	\item \textbf{\ac{DAC}} \\
		  Definiert Regeln pro Objekt (z.B. Tabellen)
	\item \textbf{\ac{MAC}} \\
		  Zusätzliche Regeln für Informationsfluss zwischen Objekt und User (mehr Sicherheit)
\end{itemize}

In SQL wird nur \ac{DAC} mit den Befehlen \texttt{grant} und \texttt{revoke} angeboten. Listing \ref{lst:grant-revoke} zeigt ein Beispiel für die Vergabe von Rechten auf eine Tabelle.

\begin{lstlisting}[caption={Rechte auf Tabelle},label=lst:grant-revoke]
-- Leserechte auf Attribut MatrNr von Tabelle prüfen für Benutzer reader
-- 'with grant option' ermöglicht dem Benutzer das Recht an weitere Benutzer zu vergeben
grant select (MatrNr)
	on prüfen
	to reader
	with grant option;
-- Weitere Rechte: delete, insert, update, references (Fremdschlüssel erstellen)

-- Rechte entziehen
-- 'cascade' entzieht auch die vom Benutzer an andere Benutzer vergebenen Rechte
revoke select (MatrNr)
	on prüfen
	from reader
	cascade;
\end{lstlisting}

\subsection{Zugriffskontrolle auf Zeilen und Spalten (views mit grants)}

Möchte man ein Recht von bestimmten Bedingungen abhängig machen, kann das mit Sichten realisiert werden. Listing \ref{lst:view-grant} zeigt als Beispiel den Tutoren alle Daten der Studenten im ersten Semester.

\begin{lstlisting}[caption={Rechte auf Zeilen und Spalten},label=lst:view-grant]
-- Sicht mit Bedingung erstellen
create view ErstSemestler as
	select MatrNr, Name
	from Studenten
	where Semester = 1;
-- Leserecht an Tutoren vergeben
grant select
	on ErstSemestler
	to tutor;
\end{lstlisting}

\subsection{Den Mechanismus von SQL-Injections erklären}

Um SQL-Abfragen flexibel zu gestalten, können vom Benutzer Parameter an die Abfrage übergeben werden. Diese Parameter können z.B. über ein HTML-Formular entgegengenommen werden. Ein Angreifer kann über dieses Formular ausführbaren SQL-Code in die Abfrage einschleusen. Deshalb darf man Benutzereingaben niemals vertrauen! Listing \ref{lst:sql-injection} zeigt eine Möglichkeit einer SQL-Injection.

\begin{lstlisting}[caption={SQL-Injection},label=lst:sql-injection]
-- Angreifer gibt als Passwort 'weissIchNichtAberEgal' or 'x' = 'x' ein
-- Da die Bedingung 'x' = 'x' immer wahr ist braucht der Angreifer das Passwort gar nicht zu wissen.
select *
from Studenten s join prüfen p on s.MatrNr = p.MatrNr
where s.Name = 'Schopenhauer' and
s.Passwort = 'weissIchNichtAberEgal' or 'x' = 'x'
\end{lstlisting}

\subsection{Zwei Lösungen beschreiben, welche vor SQL-Injections schützen}

\begin{enumerate}
	\item \textbf{Prepared Statements (Client)} \\
		  Bei Prepared Statements werden die Parameter als Platzhalter mitgegeben und die Eingaben vom Benutzer später an die Anfrage gebunden. Das ermöglicht dem \ac{DBMS} die Eingaben vor der Ausführung zu prüfen.
	\item \textbf{Stored Procedures (Server)} \\
		  Haben den selben Effekt wie Prepared Statements, sind jedoch direkt in der Datenbank (Server) gespeichert.
\end{enumerate}
