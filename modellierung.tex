\section{Lernziel M: konzeptuelle Datenbankentwürfe mit der \\ ER-Modellierungstechnik erstellen}

\subsection{Anhand der Entwurfsmethodik den Unterschied zwischen
Anforderungsspezifikation, konzeptuellem Entwurf (konzeptuelle Ebene)
und Implementationsentwurf (logische Ebene) von Datenbankmodellen
erklären (Kapitel 2.1-2.2)}

\subsubsection{Anforderungsspezifikation}

In einer Anforderungsanalyse werden die Informationsanforderungen der realen Welt bestimmt. Diese Anforderungen werden in einer Anforderungsspezifikation festgehalten. Die Anforderungsanalyse muss mit den späteren Benutzern der Anwendung zusammen durchgeführt werden.

\subsubsection{konzeptueller Entwurf}

Der konzeptuelle Entwurf wird unabhängig von der späteren Datenbank aus Anwendersicht erstellt. Als Ausgabe des konzeptuellen Entwurfs erhält man das ER-Schema.

\subsubsection{Implementationsentwurf}

Aufgrund des ER-Schemas wird der Implementationsentwurf mit dem Datenmodell der eingesetzten Datenbank erstellt. Das ER-Schema wird dann in z.B. SQL-Anweisungen umgewandelt.

\subsection{Mit der ER-Technik aufgrund einer beschriebenen Ausgangslage Entitäten, ein- und mehrstellige Relationen (inkl. Funktionalität), Attribute (inkl. Schlüssel) und Rollen ableiten, und diese in einem Diagramm darstellen (Kapitel 2.5-2.7.2) (Übung M)}

Ein ER-Modell besteht aus Entitäten und deren Beziehungen untereinander. Entitäten sind physische oder gedankliche Gegenstände, ähnlich Objekten in der OOP. Entitäten und Beziehungen können durch Attribute beschrieben werden. Ein Schlüssel ist ein Attribut das die Entität eindeutig identifiziert (z.B. Personal Nummer). Sind mehrere Schlüssel vorhanden wird ein Primärschlüssel bestimmt (ist unterstrichen). Beziehungen werden in folgende Funktionalitäten unterteilt:

\begin{itemize}
	\item \textbf{1:1-Beziehung} \\
		  Eine Entität ist genau einer anderen Entität zugeordnet (z.B. ein Mann heiratet nur eine Frau). 
	\item \textbf{1:N-Beziehung} \\
		  Eine Entität kann mehreren oder keiner anderen Entität zugeordnet werden (z.B. eine Firma beschäftigt viele Mitarbeiter).
	\item \textbf{1:1-Beziehung} \\
		  Beliebig viele Entitäten können mit anderen Entitäten in Verbindung stehen und umgekehrt.
\end{itemize}

\subsection{Fortgeschrittene Beziehungsarten in der Modellierung anwenden (Kapitel 2.8-2.10) (Übung M)}

\subsubsection{Komposition}

Eine Komposition ist eine Entität welche ohne eine andere nicht existieren kann (z.B. ein Raum in einem Gebäude). Das Rechteck einer Komposition wird mit einem doppelten Rahmen umrandet. 

\subsubsection{Generalisierung}

Bei der Generalisierung kann man gemeinsame Attribute zu einem Obertyp zusammenfassen (\verb|is a|). Die Generalisierung kann mit der Vererbung in der OOP verglichen werden.

\subsubsection{Disjunkte Spezialisierung}

Eine disjunkte Spezialisierung kann mit \emph{Fahrzeugen} erklärt werden. Von \emph{Fahrzeugen} gibt es zwei Unterentitäten: \emph{Fahrräder} und \emph{Autos}. Da ein \emph{Fahrrad} aber kein \emph{Auto} ist und ein \emph{Auto} kein \emph{Fahrrad} handelt es sich um eine disjunkte Spezialisierung.

\subsubsection{Vollständige Spezialisierung}

Bei der vollständigen Spezialisierung ist beispielsweise die Oberentität ein \emph{Spieler}. Die beiden Unterentitäten sind Sportarten: \emph{Rugby} und \emph{Fussball}. Ein \emph{Spieler} kann beide Sportarten ausüben, deshalb handelt es sich um eine vollständige Spezialisierung. 

\subsubsection{Aggregation}

Bei der Aggregation werden unterschiedliche Entitäten zu einem Gesamtobjekt zusammengefasst (\verb|part of|). Als Beispiel kann ein Fahrrad dienen das aus einem Lenker, Rädern und einer Kette besteht.