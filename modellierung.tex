\section{Lernziel M: konzeptuelle Datenbankentwürfe mit der \\ ER-Modellierungstechnik erstellen}

\subsection{Anhand der Entwurfsmethodik den Unterschied zwischen
Anforderungsspezifikation, konzeptuellem Entwurf (konzeptuelle Ebene)
und Implementationsentwurf (logische Ebene) von Datenbankmodellen
erklären (Kapitel 2.1-2.2)}

\subsubsection{Anforderungsspezifikation}

In einer Anforderungsanalyse werden die Informationsanforderungen der realen Welt bestimmt. Diese Anforderungen werden in einer Anforderungsspezifikation festgehalten. Die Anforderungsanalyse muss mit den späteren Benutzern der Anwendung zusammen durchgeführt werden.

\subsubsection{konzeptueller Entwurf}

Der konzeptuelle Entwurf wird unabhängig von der späteren Datenbank aus Anwendersicht erstellt. Als Ausgabe des konzeptuellen Entwurfs erhält man das ER-Schema.

\subsubsection{Implementationsentwurf}

Aufgrund des ER-Schemas wird der Implementationsentwurf mit dem Datenmodell der eingesetzten Datenbank erstellt. Das ER-Schema wird dann in z.B. SQL-Anweisungen umgewandelt.

\subsection{Mit der ER-Technik aufgrund einer beschriebenen Ausgangslage Entitäten, ein- und mehrstellige Relationen (inkl. Funktionalität), Attribute (inkl. Schlüssel) und Rollen ableiten, und diese in einem Diagramm darstellen (Kapitel 2.5-2.7.2) (Übung M)}



\subsection{Fortgeschrittene Beziehungsarten in der Modellierung anwenden (Kapitel 2.8-2.10) (Übung M)}

\subsubsection{Komposition}
\subsubsection{Generalisierung}
\subsubsection{Disjunkte Spezialisierung}
\subsubsection{Vollständige Spezialisierung}
\subsubsection{Aggregation}