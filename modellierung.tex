\section{Lernziel M: konzeptuelle Datenbankentwürfe mit der \\ ER-Modellierungstechnik erstellen}

\subsection{Anhand der Entwurfsmethodik den Unterschied zwischen
Anforderungsspezifikation, konzeptuellem Entwurf (konzeptuelle Ebene)
und Implementationsentwurf (logische Ebene) von Datenbankmodellen
erklären}

\subsubsection{Anforderungsspezifikation}

In einer Anforderungsanalyse werden die Informationsanforderungen der realen Welt bestimmt. Diese Anforderungen werden in einer Anforderungsspezifikation festgehalten. Die Anforderungsanalyse muss mit den späteren Benutzern der Anwendung zusammen durchgeführt werden.

\subsubsection{konzeptueller Entwurf}

Der konzeptuelle Entwurf wird unabhängig von der späteren Datenbank aus Anwendersicht erstellt. Als Ausgabe des konzeptuellen Entwurfs erhält man das ER-Schema.

\subsubsection{Implementationsentwurf}

Aufgrund des ER-Schemas wird der Implementationsentwurf mit dem Datenmodell der eingesetzten Datenbank erstellt. Das ER-Schema wird dann in z.B. SQL-Anweisungen umgewandelt.

\subsection{Mit der ER-Technik aufgrund einer beschriebenen Ausgangslage Entitäten, ein- und mehrstellige Relationen (inkl. Funktionalität), Attribute (inkl. Schlüssel) und Rollen ableiten, und diese in einem Diagramm darstellen}

Ein ER-Modell besteht aus Entitäten und deren Beziehungen untereinander. Entitäten sind physische oder gedankliche Gegenstände, ähnlich Objekten in der OOP. Entitäten und Beziehungen können durch Attribute beschrieben werden. Ein Schlüssel ist ein Attribut das die Entität eindeutig identifiziert (z.B. Personal Nummer). Sind mehrere Schlüssel vorhanden wird ein Primärschlüssel bestimmt (ist unterstrichen). Beziehungen werden in folgende Funktionalitäten unterteilt:

\begin{itemize}
	\item \textbf{1:1-Beziehung} \\
		  Eine Entität ist genau einer anderen Entität zugeordnet (z.B. ein Mann heiratet nur eine Frau). 
	\item \textbf{1:N-Beziehung} \\
		  Eine Entität kann mehreren oder keiner anderen Entität zugeordnet werden (z.B. eine Firma beschäftigt viele Mitarbeiter).
	\item \textbf{N:M-Beziehung} \\
		  Beliebig viele Entitäten können mit anderen Entitäten in Verbindung stehen und umgekehrt.
\end{itemize}

\subsection{Fortgeschrittene Beziehungsarten in der Modellierung anwenden}

\subsubsection{Komposition}

Eine Komposition ist eine Entität welche ohne eine andere nicht existieren kann (z.B. ein Raum in einem Gebäude). Das Rechteck einer Komposition wird mit einem doppelten Rahmen umrandet. 

\subsubsection{Generalisierung}

Bei der Generalisierung kann man gemeinsame Attribute zu einem Obertyp zusammenfassen (\texttt{is a}). Die Generalisierung kann mit der Vererbung in der OOP verglichen werden.

\subsubsection{Disjunkte Spezialisierung}

Eine disjunkte Spezialisierung kann mit \emph{Fahrzeugen} erklärt werden. Von \emph{Fahrzeugen} gibt es zwei Unterentitäten: \emph{Fahrräder} und \emph{Autos}. Da ein \emph{Fahrrad} kein \emph{Auto} ist und ein \emph{Auto} kein \emph{Fahrrad} handelt es sich um eine disjunkte Spezialisierung.

\subsubsection{Vollständige Spezialisierung}

Ein Beispiel: Der Basis-Entitätstyp ist \emph{Tier}. Eine Spezialisierung davon ist \emph{Hund} ein andere \emph{Katze}. Die vollständige Spezialisierung besagt, dass jedes \emph{Tier} mind. ein \emph{Hund} oder eine \emph{Katze} ist. Mit anderen Worten, es gibt kein Eintrag in der Tabelle \emph{Tier} wo nicht von \emph{Hund} oder \emph{Katze} referenziert wird. Es ist auch möglich bei der vollständigen Spezialisierung, dass das Objekt sogleich \emph{Hund} und \emph{Katze} ist, dann wäre es einfach keine disjunkte Spezialisierung.
Ein anderes Beispiel wäre: Basis-Entitätstyp \emph{Person}. Eine Spezialisierung davon ist \emph{Mitarbeiter} eine andere \emph{Kunde}. Alle Mitarbeiter und Kunden können nun erfasst werden. Das System ist so gebaut, dass nun Lieferanten einfach nur in \emph{Person} eingetragen werden. Es gibt keine konkrete Spezialisierung. Hier wäre die vollständige Spezialisierung nicht gegeben.

\subsubsection{Aggregation}

Bei der Aggregation werden unterschiedliche Entitäten zu einem Gesamtobjekt zusammengefasst (\texttt{part of}). Als Beispiel kann ein Fahrrad dienen das aus einem Lenker, Rädern und einer Kette besteht.
