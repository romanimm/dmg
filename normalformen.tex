\section{Lernziel NF: Datenmodelle normalisieren}

\subsection{Sinn und Zweck der Normalformen erklären (2.4.1)}
Normalformen sollen redundante Informationshaltung in den Dantenbankschemas vermeiden, da diese Probleme bereiten können.

\subsubsection{Redundanz}
\mygraphics{0.8\textwidth}{fig/redundanz}{Redundante Tabelle}{redundanz}
\texttt{A\#} verweist in dieser Tabelle auf die Abteilungsnummer, und die Bezeichnung auf die Abteilungsbezeichnung. Die Bezeichnung könnte ohne Informationsverlust gestrichen werden, da z.B. die Abteilungsnummer \textit{A6} sowieso immer die Finanzabteilung ist. Besser wäre es, diese Informationen auszulagern. Zudem macht es keinen Sinn, dieselbe Abteilungsbezeichnung mehrfach zu speichern, wenn eine simple Referenz darauf zeigen könnte.

\subsubsection{Anomalien}
Mit der besprochenen Redundanz treten folgende Anomalien auf:
\begin{itemize}
  \item \textbf{Einfügeanomalie} \\
  Eine neue Abteilung kann nicht ohne weiteres hinzugefügt werden, da ohne ein neuer Mitarbeiter ja keine Abteilung hinzugefügt werden kann, da die Abteilung ja nicht für sich selbst notiert wird.
  \item \textbf{Löschanomalie} \\
  Wenn die Mitarbeiter gelöscht werden, verliert man auch die Information über die Abteilungen.
  \item \textbf{Änderungsanomalie} \\
  Der Name einer Abteilung ändert. Nun muss bei jedem Mitarbeiter die Bezeichnung angepasst werden.
\end{itemize}


\subsection{Funktionale Abhängigkeit erkennen und für die Herleitung der Normalformen 1 und 2 anwenden (Kap. 2.4.2)}
\subsubsection{Funktionale Abhängigkeit}
Die Mitarbeiternummer ist eindeutig. Zu jeder Mitarbeiternummer gehört genau ein Name und einen Ort. Diese Zuordnung bestimmt die Mitarbeiternummer also eindeutig. Daher ist der Name und der Ort \emph{funktional abhängig} von der Mitarbeiternummer.

Formal ausgedrückt heisst das: Für einen Identifikationsschlüssel \(S\) und ein beliebiges Merkmal \(B\) einer Tabelle gilt die funktionale Abhängigkeit \(S \rightarrow B\).
\subsubsection{Volle funktionale Abhängigkeit}
Es kann auch zusammengesetzte Schlüssel geben, also ein Schlüssel \(S\) besteht aus \(S_{1}\) und \(S_{2}\). Ein Merkmal (also z.B. der Name) ist nur dann \emph{\textbf{voll} funktional  abhängig}, wenn das Merkmal nur aus der Zusammensetzung beider Schlüssel \(S_{1}\) und \(S_{2}\) hergeleitet werden kann. Wenn \(S_{1}\) alleine bereits genügen würde, wäre es zwar von \(S_{1}\) funktional abhängig, aber nicht \emph{voll} funktional abhängig. Klar soweit?
\mygraphics{0.6\textwidth}{fig/nf_1_2.png}{Normalform 1 und 2}{nf12}

Ein Beispiel. Wir haben die Tabelle in der ersten Normalform von der Abbildung \ref{fig:nf12}. Nun haben wir dort ein zusammengsetztes Schlüsselpaar, bestehend aus Mitarbeiternummer und Projektnummer. Nun kann z.B. eindeutig gesagt werden, dass der Name des Mitarbeiters mit Nummer 7 und Projekt 1 'Huber' ist. Also ist das Merkmal \emph{Name} funktional abhängig vom Schlüsselpaar M\# und P\# .

Nun könnte man ja auch den Namen 'Huber' alleine aus der Mitarbeiternummer herleiten, dafür bräuchte es gar keine Projektnummer! Genau deswegen ist der Name hier auch nicht \emph{voll} funktional abhängig vom Schlüsselpaar. Denn nach der Definition ist eine \emph{volle funktionale Abhängigkeit} nur dann gegeben, wenn es von Teilen des Schlüssels nicht eindeutig möglich ist, auf das entsprechende Merkmal zu schliessen.

\subsubsection{Erste Normalform}
\begin{center}
    \framebox{\textbf{In einer Spalte dürfen keine Mengen, Aufzählungen oder Wiederholungsgruppen sein.}}
\end{center}

\subsubsection{Zweite Normalform}
Erste Normalform und zusätzlich:
\begin{center}
    \framebox{\textbf{Jedes Nichtschlüsselmerkmal ist von jedem Schlüssel \textit{voll funktional abhängig}.}}
\end{center}


Deswegen wird die Tabelle aus der Abbildung \ref{fig:nf12} in der ersten Normalform aufgeteilt in eine Mitarbeitertabelle und eine Zugehörigkeitstabelle.

\subsection{Den Begriff der transitiven Abhängigkeit erkennen und für die Herleitung der Normalform 3 anwenden (Kap. 2.4.3)}
\subsubsection{Transitive Abhängigkeit}
Am besten lässt sich dies an einem Beispiel erläutern;
\mygraphics{0.6\textwidth}{fig/nf3.png}{Normalform 3}{nf3}

Wir haben den Mitarbeiter Nummer 19. Aus dieser Nummer können wir schliessen, dass unser Mitarbeiter 'Schweizer' heisst und in der Abteilung 'A6' arbeitet. Diese Merkmale sind also \emph{funktional abhängig} von der Mitarbeiternummer. Zudem ist von der Abteilungsnummer 'A6' eindeutig auf dessen Bezeichnung zu schliessen - 'Finanz'. Da wir von der Mitarbeiternummer zur Abteilungsnummer und daraus zu der Bezeichnung eindeutig schliessen können, ist die Bezeichnung ebenfalls \emph{funktional abhängig} von der Mitarbeiternummer.

Formal ausgedrückt:
\begin{equation*}
    A\rightarrow B , B\rightarrow C
\end{equation*}
Daraus folgt:
\begin{equation*}
    A\rightarrow C
\end{equation*}

Kann jetzt von der Abteilungsnummer ebenfalls eindeutig auf eine einzige Mitarbeiternummer geschlossen werden? Also gilt auch:
\begin{equation*}
    B\rightarrow A
\end{equation*}

Natürlich nicht. Somit ist die Definition für eine \emph{transitive Abhängigkeit} zwischen \(A\) und \(C\) gegeben.

Die mittlere Grafik in der Abbildung \ref{fig:nf3} zeigt dies nochmals grafisch.

\subsubsection{Dritte Normalform}
Zweite Normalform und zusätzlich:
\begin{center}
    \framebox{\textbf{Kein Nichtschlüsselmerkmal ist von irgendeinem Schlüssel transitiv abhängig}}
\end{center}

\subsection{Weitere Normalformen aufzählen:}

\subsubsection{BCNF}
Muss man nur wissen, dass es das auch gibt... Ist kompliziert.

\subsubsection{mehrwertige Abhängigkeiten}
\mygraphics{0.6\textwidth}{fig/nf4.png}{Normalform 4}{nf4}

Wir haben ein Konstrukt wie in Abbildng \ref{fig:nf4} gezeigt. Das heisst dass die Begriffe abhängig sind von der Methode und der Autor ebenfalls. Zudem gibt es mehrere Begriffe und mehrere Autoren. Deswegen trennt man das alles hübsch auf.
\subsubsection{Vierte Normalform}
Dritte Normalform, BCNF und zusätzlich:
\begin{center}
    \framebox{\textbf{Keine zwei echte und voneinander verschiedene mehrwertige Abhängigkeiten in derselben Tabelle.}}
\end{center}

\subsubsection{Verbundabhängigkeit / fünfte Normalform}
Wir haben eine Tabelle \(R\) mit Merkmalen \(A\), \(B\) und \(C\). Wir Teilen diese auf in zwei Tabellen \(R_{1}(A,B)\) und \(R_{2}(B,C)\). Wenn wir diese dann ohne Verfälschung und Informationsverluste wieder in die Tabelle \(R\) zurückführen können, dann erfüllt diese das Kriterium der fünften Normalform.